%!TEX program = xelatex
\pdfminorversion=4
\documentclass{beamer}
\usepackage{graphicx}
\usepackage{animate}
%\usepackage{amsmath}
\usepackage{bm}
\usepackage{tikz}
\usepackage{enumerate}
\usepackage{multirow}
\usepackage{setspace}
\usepackage{multimedia}
\usepackage[customcolors,shade]{hf-tikz}

\newcommand{\highlight}[2]{%
    \colorbox{#1}{$\displaystyle#2$}
}

\tikzset{
  every overlay node/.style={
    draw=black,fill=white,rounded corners,anchor=north west,
  },
}
% Usage:
% \tikzoverlay at (-1cm,-5cm) {content};
% or
% \tikzoverlay[text width=5cm] at (-1cm,-5cm) {content};
\def\tikzoverlay{%
   \tikz[baseline,overlay]\node[every overlay node]
}%

\mode<presentation>{ 
%\usetheme{UCSB-MRL2018} 
\usetheme[nototalslidenumber]{Drexel2019} 
%\usetheme[nologo]{UCSB-MRL2018} 
}

\newcommand{\backupbegin}{
    \newcounter{framenumberappendix}
    \setcounter{framenumberappendix}{\value{framenumber}}
}
\newcommand{\backupend}{
    \addtocounter{framenumberappendix}{-\value{framenumber}}
    \addtocounter{framenumber}{\value{framenumberappendix}} 
}



\title{An example of the Drexel beamer template}
\subtitle{Exploring color schemes and graphics}

\author{Joshua Lequieu}
\date{\today}
%\institute{Materials Research Laboratory\\University of California, Santa Barbara}

% math
\newcommand{\rmsub}[2]{\ensuremath{#1_{\rm #2}}}
\newcommand{\rmsubtiny}[2]{\ensuremath{#1_{\textrm{\tiny #2}}}}

\begin{document}

% ------------------------------------------------
% Title Page
% ------------------------------------------------

% make titlepage with UCSB background. Argument is the width of the title textbox
\makefancytitle{0.8\textwidth}

% make simple titlepage 
\maketitle

% ------------------------------------
%     BEGIN
% ------------------------------------
\begin{frame}{Variational Enhanced Sampling}
\begin{itemize}
\item A free energy surface, $F(\bm{s})$, associated with set of collective variables, $\bm{s}$, is defined as
\begin{equation}
    F(\bm{s}) = - (1/\beta) \log \int \mathrm{d}\bm{R} \delta(\bm{s} - \bm{s}(\bm{R}) e^{-\beta U(\bm{R})}
\end{equation}

\item Valsson and Parrinello introduce a functional of bias potential $V(\bm{s})$,
\begin{equation}
    \Omega[V] = \frac{1}{\beta} \log \frac{\int \textrm{d}\bm{s} e^{-\beta[F(\bm{s}) + V(\bm{s})]}} {\int \textrm{d}\bm{s} e^{-\beta F(\bm{s})}} + \int \textrm{d}\bm{s} p(\bm{s}) V(\bm{s})
\end{equation}
where $p(\bm{s})$ is an arbitrary probability distribution (normalized).
\\\textcolor{red}{$\Rightarrow$ $\Omega[V]$ is convex!}

\item The potential which renders $\Omega[V]$ stationary is (to within a constant):
\begin{equation}
    V(\bm{s}) = - F(\bm{s}) - (1/\beta) \log p(\bm{s})
\end{equation}

\end{itemize}

\tiny \hfill Valsson, Parrinello. (2014) PRL 113,090601 
\end{frame}

\begin{frame}{Variational Enhanced Sampling}
\begin{enumerate}
\item Write bias potential $V(\bm{s};\bm{\alpha})$ as function of variational parameters $\bm{\alpha} = (\alpha_1, \alpha_2, ..., \alpha_K)$

\item Minimize $\Omega[V(\bm{\alpha})]$ with respect to $\bm{\alpha}$ using the gradient $\Omega^\prime(\bm{\alpha})$ and Hessian $\Omega^{\prime\prime}(\bm{\alpha})$,
\begin{equation}
\frac{\partial \Omega}{\partial \alpha_i} = - \biggl< \frac{\partial V(\bm{s};\bm{\alpha})}{\partial \alpha_i} \biggr>_{V(\bm{\alpha})}
                                            + \biggl< \frac{\partial V(\bm{s};\bm{\alpha})}{\partial \alpha_i} \biggr>_{p}
\end{equation}

{\scriptsize
\begin{align}
\frac{\partial^2 \Omega(\bm{\alpha})}{\partial \alpha_j \partial \alpha_i} = & \beta \mathrm{Cov} \left[\frac{\partial V(\bm{s};\bm{\alpha})}{\partial \alpha_j}; \frac{\partial V(\bm{s}; \bm{\alpha})}{\partial \alpha_i} \right]_{V(\bm{\alpha})} %\nonumber \\
               & - \biggl<\frac{\partial^2 V(\bm{s};\bm{\alpha})}{\partial \alpha_j \partial \alpha_i} \biggr>_{V(\bm{\alpha})} 
                 + \biggl<\frac{\partial^2 V(\bm{s};\bm{\alpha})}{\partial \alpha_j \partial \alpha_i} \biggr>_{p} 
\end{align}
}
where $\langle ... \rangle_{V(\bm{\alpha})}$ and $\langle ... \rangle_{p(\bm{s})}$ are the expectation value obtained in a biased simulation with potential $V(\bm{s};\bm{\alpha})$ and distribution $p(\bm{s})$, respectively
\item Use stochastic gradient descent-based algoritm to iterate $\bm{\alpha}$
\\\textcolor{UCSBaqua}{$\Rightarrow$ In practice, only the diagonal of the Hessian is used during optimization}

\end{enumerate}

\tiny \hfill Valsson, Parrinello. (2014) PRL 113,090601 
\end{frame}

\begin{frame}{Block copolymers are awesome}{They can assemble into so many different morphologies}
    \begin{block}{Block title}
    Block content in here
    \end{block}
\end{frame}

\begin{frame}{%
This title is approximately 46 characters. ABC}
\end{frame}
\begin{frame}{%
This title is approximately 47 characters. ABCD}{Note no logo!}
\end{frame}


% ---------------------
%       APPENDIX
% ---------------------
%\appendix 
%\backupbegin
%
%\begin{frame}{Appendix}
%\end{frame}
%
%\backupend

\end{document}
